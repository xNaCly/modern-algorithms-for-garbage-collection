\chapter{Comparison with other Memory Management Techniques}

\section{Manual Memory Management}

\section{Lifetimes and Borrow Checking}

The desire for the performance of manual memory management and the safety of garbage collection has led to the development
of a new memory management technique called \textit{lifetimes and borrow checking}.
The main idea behind this technique is that the corresponding \texttt{free} calls for heap memory can be automatically inserted
at compile time by the compiler, if the compiler can prove that the memory is no longer needed.
When a variable is no longer needed, it is said to have reached the end of its \textit{lifetime} hence the name of the technique.

This memory management technique was first introduced in the Rust programming language \cite[1. Introduction]{rust_borrow_formalism_2021}.
Because Rust was the first language to implement this concept, the examples in this section will be written in Rust.

\subsubsection{Ownership}

\subsubsection{Borrowing}

\subsubsection{Long-lived variables using reference counters}
