\chapter{Garbage collected Programming Languages}

In this chapter, the used garbage collection implementations of two programming
languages are presented. These are implementations of the theoretical concepts
presented in \autoref{sec:overview}

\section{Go}
\section{Java}

Java by default uses a generational garbage collector as introduced in \autoref{sec:gc_generational}.
This garbage collector is called Garbage First (G1) and was introduced in Java 9 \cite{java_gc_comparison_2018}.
Before that, Java used various types of mark and sweep collectors \cite{java_available_gcs}.

Beyond those there are many more garbage collectors available for Java that can
be used by specifying them as a command line argument to the JVM.
These are not relevant for this writing, as they are not used by default.
Nonetheless these can be very useful when wanting to use a garbage collector
tuned to a specific use case.

Contrary to the theoretical concept of a generational garbage collector,
the memory areas for each generation in G1GC are not continuous in memory.
Instead G1GC uses a heap divided into regions usually 1 MB - 32 MB in size.
Each region is assigned to one of the generations.
These generations are called Eden, Survivor and Old. \cite{java_g1_getting_started}

% TODO: compacting, shendoah
% TODO: configurable gc pauses (length)

