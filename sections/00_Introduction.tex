\section{Introduction}

Garbage collection refers to the process of automatically managing
heap allocated memory on behalf of the running process by
identifying parts of memory that are no longer needed. \cite[Introduction]{go_gcguide_2022}
This is often performed by the runtime of a programming language while the
program is executing. \cite[Introduction]{go_spec_2023}

Most programming languages allocate values with static
lifetimes\footnote{variable available for the whole runtime of the program
\cite[Abstract]{static-dynamic-scope_tanter_2009}} in main memory along with
the executable code. Values that are alive for a certain scope are allocated
using the call stack\footnote{stores information about running subroutines /
functions\cite[2.2 Call Stacks]{call-stack_mcmaster-memon_2006}} without
requiring dynamic allocation. These Variables can't escape the scope they were
defined in and must be dynamically allocated if accessing them outside of
their scope is desired.

This requires the programmer to allocate and deallocate these
variables to prevent memory leaks\footnote{allocated no longer needed memory
not deallocated \cite[1.2.1 A Practical Object Ownership
Model]{practical_heine-lam_2003}} provided the programming language does not
perform garbage collection. 

\begin{listing}[H] 
    \begin{minted}{c} 
#include <stdio.h>
#include <stdlib.h>
typedef struct { 
    char *name;
    double age; 
} Person;

Person *new_person(char *name, double age) {
  Person *p = malloc(sizeof(Person));
  p->age = age, p->name = name;
  return p;
}
// [...]
    \end{minted}
    \caption{C heap allocation}
    \label{code:c_heap_alloc_example}
\end{listing}

Listing \autoref{code:c_heap_alloc_example} showcases a possible use case for
dynamic memory allocation. The \texttt{Person} structure is filled with values
defined in the parameters of the \texttt{new\_person} function. This structure,
if stack allocated, would not live longer than the scope of the
\texttt{new\_person} function, thus rendering this function useless. To create
and use a \texttt{Person} structure outside of its scope, the structure has to
be dynamically allocated via the \texttt{malloc} function defined in the
\mintinline{c}{#include <stdlib.h>} header.

\begin{listing}[H] 
    \begin{minted}{c} 
// [...]
int main(void) {
    for(int i = 0; i < 1e5; i++) {
        Person *p = new_person("max musterman", 89);
    }
    return EXIT_SUCCESS;
}
    \end{minted}
    \caption{C heap allocation with memory leakage}
    \label{code:c_heap_memory_leak_example}
\end{listing}

See listing \autoref{code:c_heap_memory_leak_example} for an example of a
memory leak. Here the program creates $1*10^5$ \texttt{Person} structures
using the \texttt{new\_person} function allocating each one on the heap but
not releasing their memory after the iteration ends and therefore rendering
the reference to them inaccessible, which generally defines a memory leakage
this programming error can lead to abnormal system behaviour and excessive RAM
consumption in long lived applications
\cite[Description]{owasp-memoryleak_unknown_unknown}. The definitive solution
for memory leaks is determining leaking variables and freeing them, see
listing \autoref{code:c_heap_memory_leak_example_fixed}.

\begin{listing}[H] 
    \begin{minted}{c} 
// [...]
int main(void) {
    for(int i = 0; i < 1e5; i++) {
        Person *p = new_person("max musterman", 89);
        free(p);
    }
    return EXIT_SUCCESS;
}
    \end{minted}
    \caption{C heap allocation without memory leakage}
    \label{code:c_heap_memory_leak_example_fixed}
\end{listing}

Another potential issue with manual memory management is accessing already
released memory classified as \textit{use-after-free} errors
\cite{owasp-use-after-free_unkown_unknown}. Consider the modified example in
listing \autoref{code:c_heap_memory_use_after_free_example} showcasing value
access of a \texttt{Person} structure after its memory has already been
released.

\begin{listing}[H] 
    \begin{minted}{c} 
// [...]
int main(void) {
    for(int i = 0; i < 1e5; i++) {
        Person *p = new_person("max musterman", 89);
        free(p);
        printf("Person{name: '%s', age: %f}\n", p->name, p->age);
    }
    return EXIT_SUCCESS;
}
    \end{minted}
    \caption{C heap allocation with freed memory access}
    \label{code:c_heap_memory_use_after_free_example}
\end{listing}

The example in listing \autoref{code:c_heap_memory_use_after_free_example}
results in undefined behaviour
\cite[Description]{owasp-use-after-free_unkown_unknown} and could cause
crashes if memory the program can not legally access is accessed, could cause
memory corruption if the memory region pointed to contains data after the
previous data has been released or could be exploited to inject data into the
application \cite[Consequences]{owasp-use-after-free_unkown_unknown}.

\begin{listing}[H] 
    \begin{minted}{c} 
// [...]
void *free_person(Person *p) {
  free(p);
  return NULL;
}

int main(void) {
    for(int i = 0; i < 1e5; i++) {
        Person *p = new_person("max musterman", 89);
        p = free_person(p);
        if(p == NULL) continue;
        printf("Person{name: '%s', age: %f}\n", p->name, p->age);
    }
    return EXIT_SUCCESS;
}
    \end{minted}
    \caption{C heap allocation without freed memory access}
    \label{code:c_heap_memory_use_after_free_example_fixed}
\end{listing}

A common resolution for this issue is setting a pointer to \texttt{NULL} via
\mintinline{c}{p = NULL} and checking if the pointer is \texttt{NULL} before
accessing it (see listing
\autoref{code:c_heap_memory_use_after_free_example_fixed}) \cite[Related
Controls]{owasp-use-after-free_unkown_unknown}. 

Garbage collection manages dynamically allocated memory for the programmer,
therefore issues such as memory leakages and accessing released memory can be
prevented by not exposing capabilities for manual memory management. A
language such as golang contains a garbage collector
\cite[Introduction]{go_spec_2023} enabling autoamtically releasing no longer
used memory blocks, as shown in listing \autoref{code:go_heap_alloc_example}.

\begin{listing}[H] 
    \begin{minted}{go} 
package main

import "fmt"

type Person struct {
    Name string
    Age  float64
}

func new_person(name string, age float64) *Person {
    return &Person{name, age}
}

func main() {
    for i := 0; i < 1e5; i++ {
        p := new_person("max musterman", 89)
        fmt.Printf("Person{name: %q, age: %f}\n", p.Name, p.Age)
    }
}
    \end{minted}
    \caption{Go allocation example}
    \label{code:go_heap_alloc_example}
\end{listing}

The garbage collector in listing \autoref{code:go_heap_alloc_example}
automatically deallocates the result of \texttt{new\_person} upon it leaving
the scope of the loop iteration it was called in.
