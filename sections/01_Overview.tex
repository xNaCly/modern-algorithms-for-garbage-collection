\chapter{Garbage Collection}

As introduced before (see \autoref{sec:introduction}) the process of garbage
collection is required by many programming languages via their specification,
as is the case with \textit{Java} \cite[Chapter 1.
Introduction]{java_language_spec_2023} and \textit{Go}
\cite[Introduction]{go_spec_2023}. The \textit{Go} programming language
specification however does not include specifics around the implementation of
its garbage collection \cite[Introduction]{go_gcguide_2022}. The \textit{Go}
standard tool chain provides a runtime library included in all executables
created by the \textit{Go} compiler. This library contains the garbage
collector \cite[Introduction]{go_gcguide_2022}.

Garbage collection as a whole is an umbrella term for
different concepts, algorithms and ideas. This chapter includes the
differentiation between these and thereby introduces terms necessary for
understanding the following chapters. 

\section{Scope}
\label{sec:scope}

The scope of garbage collection refers to the variables, resources and memory
areas it manages. Garbage collection is generally responsible for managing
already allocated memory, either by the programmer or the libraries /
subroutines the programmer uses \cite[Abstract]{learned-gc_2020}. The
aforementioned can be cumulated to heap allocated memory or dynamically
allocated memory. This represents the purview of a garbage collector \cite[1
Introduction]{age-based-gc_1999}. The listing \autoref{code:java_gc_variables}
showcases variables that will be garbage collected upon the scope of the
\texttt{GarbageCollected.main()} function ends.

\begin{listing}[H] 
    \begin{minted}{java} 
class GarbageCollected {
    public static void main(String[] args) {
        var test1 = new Test();
        var test2 = new Test();
    }
    public Test() {}
}
    \end{minted}
    \caption{Java variables managed by the garbage collector}
    \label{code:java_gc_variables}
\end{listing}

The areas not managed by the garbage collector and thus not in the scope of
this paper are open resources requiring being closed by the consumer (such as
sockets or \texttt{java.util.Scanner} \cite[close]{java-util-scanner}) and
stack allocated variables as well as statically allocated variables. The
listing \autoref{code:java_non_gc_variables} displays a variety of variables
not garbage collected due to all of them being stack allocated primitive types
\cite[4.2. Primitive Types and Values]{java_language_spec_2023}.

\begin{listing}[H] 
    \begin{minted}{java} 
class NotGarbageCollected {
    static int integer = 5;
    public static void main(String[] args) {
        byte newline = 0x1A;
        double pi = 3.1415;
        char a = 'a';
    }
}
    \end{minted}
    \caption{Java variables not managed by the garbage collector}
    \label{code:java_non_gc_variables}
\end{listing}

\section{Strategies}

Garbage collection can be implemented using a variety of strategies, each
differing in their code complexity, RAM/CPU usage and execution speed \cite[4.3
Benchmarks]{gc-performance_2004} \cite[Motivation and Historical
Perspective]{gc-hardware_2018}.

\subsection{Tracing}

Most commonly the term garbage collection is used to refer to tracing garbage
collection. This strategy of automatically managing memory is a common way of
implementing garbage collection. Tracing is defined as determining which
objects should be deallocated. This is done by tracing which of the currently
allocated objects are accessible via linked references. Accessible objects are
marked as alive. Memory regions not accessible via this list are not marked and
therefore considered to be unused memory and are deallocated. \cite[Garbage
Collection Background]{gc-hardware_2018}

Programming languages such as \textit{Java} \cite[2.2 Full GC
algorithm]{java_gc_comparison_2019}, \textit{Go} \cite[Tracing Garbage
Collection]{go_gcguide_2022} and \textit{Ocaml} \cite[Garbage Collection,
Reference Counting, and Explicit Allocation]{ocmal_gc_unknown} use this
strategy for deallocating unused memory regions.

\subsubsection{Categorizing memory}
\label{sec:categorizing_memory}

Objects\footnote{Dynamically allocated memory region containing one or more
values \cite[Tracing Garbage Collection]{go_gcguide_2022}} are categorised as
reachable or alive if they are referenced by at least one variable in the
currently running program, see \autoref{code:java_memory_categories_example}
for a visualisation. This includes references from other reachable objects. As
introduced before, the definition of tracing garbage collection includes
determining whether or not objects are reachable. In the paragraph above, this
reachability is defined. This definition does not include the objects the
tracing garbage collector refers to as \textit{root}-objects
\cite[Preliminaries: Heap Depth and Tracing]{tracing-gc_barabash_2010}.
root-objects are defined as generally accessible, such as local variables,
parameters and global variables.\footnote{As introduced in \autoref{sec:scope}:
variables on the call stack or static variables} Root-objects are used as a
starting point for tracing allocated objects \cite[Preliminaries: Heap Depth
and Tracing]{tracing-gc_barabash_2010}.\\

In \autoref{code:java_memory_categories_example}, both values initially
assigned to \texttt{x} and \texttt{y} in the \texttt{Main.main} function are
considered inaccessible due to the reassignment of \texttt{x} and \texttt{y}
in the following lines. The value of the variable \texttt{z} in the
\texttt{Main.f} function is considered inaccessible once the scope of the
function ends, when the variable \texttt{z} is dropped from the call stack -
rendering its value inaccessible. 

\begin{listing}[H] 
    \begin{minted}{Java} 
public class Main {
    public static void main(String[] args) {
        var x = new Object();
        x = new Object();
        var y = new Object();
        y = new Object();
        Main.f();
    }

    private static void f() {
        var z = new Object();
    }
}
    \end{minted}
    \caption{Java example for accessible and inaccessible memory}
    \label{code:java_memory_categories_example}
\end{listing}


\subsubsection{Implementations}

As introduced before the main idea behind tracing garbage collection is to
trace the memory set\footnote{Virtual memory the program makes use of}. Garbage
collection is often performed in cycles. Cycles are triggered when certain
conditions are met, such as the program running out of memory and therefore not
being able to satisfy an allocation request or the cycles are ran on a
predefined interval. The process of tracing memory and deallocating memory
require separation, they are therefore often split into different garbage
collection cycles. The following concepts and implementation details can be and
are generally intertwined in modern garbage collectors \cite[The GC
cycle]{go_gcguide_2022} \cite{ocmal_gc_unknown}.

\paragraph{Mark and Sweep}

Garbage collectors using the \textit{mark and sweep}-concept traverse the
object graph\footnote{Objects and pointers to objects} starting from the
root-objects, therefore satisfying the definition of a tracing garbage
collector, as presented in \autoref{sec:categorizing_memory}. The main detail
of the mark and sweep technique is marking each encountered object of the
object graph as alive. This stage of the process is referred to as
\textit{marking}. The stage defined as \textit{sweeping} entails walking over
the memory on the heap and deallocating all non marked objects \cite[Tracing
Garbage Collection]{go_gcguide_2022}.

\paragraph{Generational}

Generational garbage collection is based on the empirical observation that
recently allocated objects are most likely to be inaccessible
quickly\footnote{Generally known as \textit{infant mortality} or
\textit{generational hypothesis}}. Objects are differentiated into generations,
this is often implemented by using separate memory regions for different
generations. Upon filling a generations memory region its objects are being
traced by using the older generation as roots, this usually results in most
objects of the generation being deallocated. The remaining objects are moved
into the older generations memory region \cite[2 Age-based Garbage
Collection]{age-based-gc_1999}. This technique results in fast incremental
garbage collection, considering one memory region at a time is required to be
collected. \cite[3 Benchmarks]{age-based-gc_1999}

\paragraph{Stop the world}

Stop the world garbage collector refer to the process of halting the execution
of the program for running a garbage collection cycle. Therefore guaranteeing
that no new objects are allocated or becoming unreachable while performing the
garbage collection cycle. The main advantage of this implementation approach is
that it introduces less code complexity while being faster than the previously
introduced incremental garbage collection \cite[5. The Garbage Collection
Algorithms]{gc-multiprocess_1975}.

\subsection{Reference Counting}

Reference counting garbage collection is defined as each object keeping track
of the amount of references made to it. This reference counter is incremented
for each created reference and decremented for each destroyed reference. Once
the counter reaches 0 the object is no longer considered reachable and
therefore deallocated \cite[2.2 Precise Reference
Counting]{gc-references_2021}. \\

In contrast to the previously introduced tracing garbage collection this
approach promises that objects are immediately deallocated once their last
reference is destroyed. Due to the reference count being attached to their
respective objects this strategy is CPU cache friendly \cite[1.
Introduction]{gc-references_2021}.

\subsubsection{Disadvantages}

Contrary to the aforementioned, reference counting garbage collection has
several disadvantages to tracing \cite[2. Overview]{gc-references_2021}. These
can be mitigated via sophisticated algorithms.

\paragraph{Memory usage}
Reference counting requires attaching a reference counter onto allocated
objects, thus increasing the overall memory footprint proportionally to the
amount of allocated objects and a reference counter for each object.

\begin{align*}
n &:= \textrm{Amount of Objects}\\
m &:= \textrm{Object size} \\
r &:= \textrm{Reference counter size}
\end{align*}

Memory footprint without reference counting:
\begin{align*}
n \cdot m 
\end{align*}

Memory footprint with reference counting:
\begin{align*}
(n \cdot m) + (n \cdot r)
\end{align*}

\paragraph{Cycles}
\paragraph{Increment and Decrement Workload}
\paragraph{Thread safety}
\paragraph{Real-time performance}

\subsection{Escape Analysis}
% https://en.wikipedia.org/wiki/Garbage_collection_%28computer_science%29#Escape_analysis
% https://en.wikipedia.org/wiki/Escape_analysis
