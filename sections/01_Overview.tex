\section{Garbage Collection}

As introduced before (see \autoref{sec:introduction}) the process of garbage
collection is required by many programming languages via their specification,
as is the case with \textit{Java} \cite[Chapter 1.
Introduction]{java_language_spec_2023} and \textit{Go}
\cite[Introduction]{go_spec_2023}. The \textit{Go} programming language
specification however does not include specifics around the implementation of
its garbage collection \cite[Introduction]{go_gcguide_2022}. The \textit{Go}
standard tool chain provides a runtime library included in all executables
created by the \textit{Go} compiler. This library contains the garbage
collector \cite[Introduction]{go_gcguide_2022}.

% TODO: cite
Garbage collection as a whole is an umbrella term for
different concepts, algorithms and ideas. This chapter includes the
differentiation between these and thereby introduces terms necessary for
understanding the following chapters. 

\subsection{Scope}

The scope of garbage collection refers to the variables, resources and memory
areas it is used to manage. 

% TODO: cite
Garbage collection is generally responsible for managing already by the
programmer or the libraries / subroutines the programmer uses allocated memory.
The aforementioned can be cumulated as heap allocated memory or dynamically
allocated memory and represents the purview of a garbage collector.

The areas not managed by the garbage collector and thus not in the scope of
this paper are open resources requiring being closed by the consumer (such as
sockets or \texttt{java.util.Scanner} \cite[close]{java-util-scanner}) and
stack allocated variables as well as statically allocated variables.

\subsection{Strategies}
