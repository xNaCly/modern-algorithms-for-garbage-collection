\chapter{Garbage Collection}

As introduced before (see \autoref{sec:introduction}) the process of garbage
collection is required by many programming languages via their specification,
as is the case with \textit{Java} \cite[Chapter 1.
Introduction]{java_language_spec_2023} and \textit{Go}
\cite[Introduction]{go_spec_2023}. The \textit{Go} programming language
specification however does not include specifics around the implementation of
its garbage collection \cite[Introduction]{go_gcguide_2022}. The \textit{Go}
standard tool chain provides a runtime library included in all executables
created by the \textit{Go} compiler. This library contains the garbage
collector \cite[Introduction]{go_gcguide_2022}.

Garbage collection as a whole is an umbrella term for
different concepts, algorithms and ideas. This chapter includes the
differentiation between these and thereby introduces terms necessary for
understanding the following chapters. 

\section{Scope}
\label{sec:scope}

The scope of garbage collection refers to the variables, resources and memory
areas it manages. Garbage collection is generally responsible for managing
already allocated memory, either by the programmer or the libraries /
subroutines the programmer uses \cite[Abstract]{learned-gc_2020}. The
aforementioned can be cumulated to heap allocated memory or dynamically
allocated memory. This represents the purview of a garbage collector \cite[1
Introduction]{age-based-gc_1999}. The listing \autoref{code:java_gc_variables}
showcases variables that will be garbage collected upon the scope of the
\texttt{GarbageCollected.main()} function ends.

\begin{listing}[H] 
    \begin{minted}{java} 
class GarbageCollected {
    public static void main(String[] args) {
        var test1 = new Test();
        var test2 = new Test();
    }
    public Test() {}
}
    \end{minted}
    \caption{Java variables managed by the garbage collector}
    \label{code:java_gc_variables}
\end{listing}

The areas not managed by the garbage collector and thus not in the scope of
this paper are open resources requiring being closed by the consumer (such as
sockets or \texttt{java.util.Scanner} \cite[close]{java-util-scanner}) and
stack allocated variables as well as statically allocated variables. The
listing \autoref{code:java_non_gc_variables} displays a variety of variables
not garbage collected due to all of them being stack allocated primitive types
\cite[4.2. Primitive Types and Values]{java_language_spec_2023}.

\begin{listing}[H] 
    \begin{minted}{java} 
class NotGarbageCollected {
    static int integer = 5;
    public static void main(String[] args) {
        byte newline = 0x1A;
        double pi = 3.1415;
        char a = 'a';
    }
}
    \end{minted}
    \caption{Java variables not managed by the garbage collector}
    \label{code:java_non_gc_variables}
\end{listing}

\section{Strategies}

Garbage collection can be implemented using a variety of strategies, each
differing in their code complexity, RAM/CPU usage and execution speed.
\cite[4.3 Benchmarks]{gc-performance_2004}

\subsection{Tracing}
% TODO: cite whole section

Most commonly the term garbage collection is used to refer to tracing garbage
collection. This strategy of automatically managing memory is a common way of
implementing garbage collection. Tracing is defined as determining which
objects should be deallocated. This is done by tracing which of the currently
allocated objects are accessible via linked references. Accessible objects are
marked as alive. Memory regions not accessible via this list are not marked and
therefore considered to be unused memory and are deallocated.

\subsubsection{Categorizing memory}
\label{sec:categorizing_memory}
% TODO: cite whole section

Objects\footnote{Dynamically allocated memory region containing one or more
values \cite[Tracing Garbage Collection]{go_gcguide_2022}} are categorised as
reachable or alive if they are referenced by at least one variable in the
currently running program, see \autoref{code:java_memory_categories_example}
for a visualisation. This includes references from other reachable objects. As
introduced before, the definition of tracing garbage collection includes
determining whether or not objects are reachable. In the paragraph above, this
reachability is defined. This definition does not include the objects the
tracing garbage collector refers to as \textit{root}-objects
\cite[Preliminaries: Heap Depth and Tracing]{tracing-gc_barabash_2010}.
root-objects are defined as generally accessible, such as local variables,
parameters and global variables.\footnote{As introduced in \autoref{sec:scope}:
variables on the call stack or static variables} Root-objects are used as a
starting point for tracing allocated objects \cite[Preliminaries: Heap Depth
and Tracing]{tracing-gc_barabash_2010}.\\

In \autoref{code:java_memory_categories_example}, both values initially
assigned to \texttt{x} and \texttt{y} in the \texttt{Main.main} function are
considered inaccessible due to the reassignment of \texttt{x} and \texttt{y}
in the following lines. The value of the variable \texttt{z} in the
\texttt{Main.f} function is considered inaccessible once the scope of the
function ends, when the variable \texttt{z} is dropped from the call stack -
rendering its value inaccessible. 

\begin{listing}[H] 
    \begin{minted}{Java} 
public class Main {
    public static void main(String[] args) {
        var x = new Object();
        x = new Object();
        var y = new Object();
        y = new Object();
        Main.f();
    }

    private static void f() {
        var z = new Object();
    }
}
    \end{minted}
    \caption{Java example for accessible and inaccessible memory}
    \label{code:java_memory_categories_example}
\end{listing}


\subsubsection{Implementations}

As introduced before the main idea behind tracing garbage collection is to
trace the memory set\footnote{Virtual memory the program makes use of}. Garbage
collection is often performed in cycles. Cycles are triggered when certain
conditions are met, such as the program running out of memory and therefore not
being able to satisfy an allocation request or the cycles are ran on a
predefined interval. The process of tracing memory and deallocating memory
require separation, they are therefore often split into different garbage
collection cycles \cite[The GC cycle]{go_gcguide_2022}.

\paragraph{Mark and Sweep}

Garbage collectors using the \textit{mark and sweep}-concept traverse the
object graph\footnote{Objects and pointers to objects} starting from the
root-objects, therefore satisfying the definition of a tracing garbage
collector, as presented in \autoref{sec:categorizing_memory}. The main detail
of the mark and sweep technique is marking each encountered object of the
object graph as alive. This stage of the process is referred to as
\textit{marking}. The stage defined as \textit{sweeping} entails walking over
the memory on the heap and deallocating all non marked objects \cite[Tracing
Garbage Collection]{go_gcguide_2022}.

\paragraph{Generational}
\paragraph{Stop the world}
\subsubsection{Performance}

\subsection{Reference Counting}
\subsection{Escape Analysis}
