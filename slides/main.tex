\documentclass{beamer}
\usepackage[ngerman]{babel}
\usepackage[utf8]{inputenc}
\usepackage{graphicx}
\graphicspath{{./assets/}}

\title{Modern Algorithms for Garbage Collection}
\subtitle{Outlining modern algorithms for garbage collection on the examples of Go and Java}
\date[2023]{2023}
\author{9525469, 6424019} 

\institute[DHBW Mosbach]{DHBW Mosbach}

\date{\today}

\setbeamertemplate{navigation symbols}{}
\setbeamertemplate{footline}[frame number]

\begin{document}
    
    \frame{\titlepage}

    \tableofcontents
    
	\section{Einleitung}
        \begin{frame}
            \frametitle{Garbage collection}
        \end{frame}

    \section{Strategien}
        \begin{frame}
            \frametitle{Strategien}
        \end{frame}

        \subsection{Mark \& Sweep}
            \begin{frame}
                \frametitle{Strategien - Mark \& Sweep}
            \end{frame}

        \subsection{Generational garbage collection}
            \begin{frame}
                \frametitle{Strategien - Generational garbage collection}
                Optimierung basierend auf der Beobachtung, dass die meisten Objekte nur kurzlebig sind (Infant Mortality)\\
                Aufteilung in drei Speicherbereiche aufgeteilt:
                \begin{itemize}
                    \item Young, bestehend aus Eden und Survivor Space
                    \item Old
                    \item Permanent
                \end{itemize}
                % Unterteilung des Heaps in die Generationen:
                % https://www.oracle.com/technetwork/tutorials/tutorials-1876574.html
                % https://www.oracle.com/webfolder/technetwork/tutorials/obe/java/G1GettingStarted/images/HeapStructure.png
                \includegraphics[width=\textwidth]{images/GenerationalGCHeapStructure.png}
            \end{frame}

            \begin{frame}
                \frametitle{Strategien - Generational garbage collection}
                \begin{itemize}
                    \item Eden
                    \begin{itemize}
                        \item Alle neuen Objekte werden hier allokiert
                        \item Objekte welche eine gewisse Zeit überleben werden in den Survivor Space verschoben
                    \end{itemize}
                    \item Survivor
                    \begin{itemize}
                        \item Hier sind Objekte welche eine gewisse Zeit überlebt haben
                        \item Objekte welche eine längere Zeit überlebt haben werden in den Old Space verschoben
                    \end{itemize}
                    \item Old
                    \begin{itemize}
                        \item Hier sind Objekte welche schon länger existieren und es unwahrscheinlich ist, dass sie bald gelöscht werden
                    \end{itemize}
                \end{itemize}
                Eden wird aufgrund von Infant Mortality häufig collected, ist aber klein da lang lebende Objekte nicht dort sind.
            \end{frame}

        \subsection{Reference counting}
            \begin{frame}
                \frametitle{Strategien - Reference counting}
            \end{frame}

    \section{Programmiersprachen}
        \begin{frame}
            \frametitle{Programmiersprachen}

            \begin{itemize}
                \item Viele Sprachen mit und ohne GC
                \item Algorithmen, Implementierungen und Performance sehr unterschiedlich
                \item High level sprachen eher mit GC, low level eher ohne
            \end{itemize}
        \end{frame}

        \subsection{Ohne Garbage collection}
            \begin{frame}
                \frametitle{Programmiersprachen - Ohne gc}

                \begin{itemize}
                    \item C
                    \begin{itemize}
                        \item manuelles memory managment
                    \end{itemize}
                    \item C++
                    \begin{itemize}
                        \item manuelles memory managment
                        \item reference counting on demand
                    \end{itemize}
                    \item Rust
                    \begin{itemize}
                        \item borrow checker
                        \item reference counting on demand
                    \end{itemize}
                \end{itemize}
            \end{frame}

        \subsection{Mit Garbage collection}
            \begin{frame}
                \frametitle{Programmiersprachen - Mit gc}
                \begin{itemize}
                    \item Go
                    \begin{itemize}
                        \item escape analysis
                        \item mark \& sweep
                    \end{itemize}
                    \item Java
                    \begin{itemize}
                        \item mehrere garbage collectoren
                        \item generational standardmäßig
                        \item JIT, Bytecode vm (JVM)
                    \end{itemize}
                    \item Python
                    \begin{itemize}
                        \item Reference counting
                        \item Erkennung von Referenzzyklen
                    \end{itemize}
                    \item JavaScript
                    \begin{itemize}
                        \item generational
                        \item JIT, Bytecode vm (V8)
                    \end{itemize}
                \end{itemize}
            \end{frame}

    \section{Implementierungen}
        \begin{frame}
            \frametitle{Garbage collection Implementierungen}
        \end{frame}

        \subsection{Go}
            \begin{frame}
                \frametitle{Implementierungen - Go}
            \end{frame}

        \subsection{Java}
            \begin{frame}
                \frametitle{Implementierungen - Java}
            \end{frame}

    \section{Performance}
        \begin{frame}
            \frametitle{Performance}

            Diverse Kriterien möglich, hier beschränkt auf:

            \begin{itemize}
                \item Speicherverbrauch: stärke RAM-Intensivität
                \item Latenz: Umfang Stoppzeiten des Programs
                \item Sicherheit: Speicherzugriffsicherheit
                \item Nutzbarkeit: Komplexität der Strategien
            \end{itemize}
        \end{frame}

        \subsection{Speicherverbrauch}
            \begin{frame}
                \frametitle{Performance - Speicherverbrauch}
            \end{frame}

        \subsection{Latenz}
            \begin{frame}
                \frametitle{Performance - Latenz}
            \end{frame}

        \subsection{Sicherheit}
            \begin{frame}
                \frametitle{Performance - Sicherheit}
            \end{frame}

        \subsection{Nutzbarkeit}
            \begin{frame}
                \frametitle{Performance - Nutzbarkeit}
            \end{frame}

\end{document}

